\documentclass[12pt,a4paper]{article}
%le préambule

\usepackage[T1]{fontenc}
\usepackage[francais]{babel}
\usepackage{url} % permet l'utilisation de la balise url pour les liens internet
\usepackage{graphicx} % permet l'utilisation des images
\usepackage{tikz}


\title{Rapport de projet - DEV WEB - Gouraya Quiz} % titre de l'article
\author{MEZIANE Ghilas MENAA Abderrezak} % auteur de l'article
\date{10/05/2021} % date de l'article

%document principal
\begin{document}

\maketitle

\tableofcontents

\newpage

\section{Gouraya Quiz - Le concept}
\subsection{Qu'est-ce que le site Gouraya Quiz ?}
\begin{center}
   \includegraphics[width=350pt]{GourayaQuiz.png}
  \end{center}
 Gouraya Quiz est un site de création et de mise à dispositions de questionnaires de type QCM , offrant la possibilité aux enseignants de tous les secteurs ainsi qu'a leurs étudiants d'intéragir via des questionnaires, simple à élaborer , et simple à répondre. Deux interfaces sont disponibles, une interface Enseignant et une interface Étudiant. 
  
\subsection{Les technologies utilisées:}
  Les technologies principales utilisées à l'élaboration de ce site web sont : 
\begin{itemize}
\item php
\item mysql
\item javascript
\item html5
\item css3
\end{itemize}
\begin{center}
\includegraphics[width=50pt]{php.png}
\includegraphics[width=100pt]{mysql.png}
\includegraphics[width=100pt]{html_css_js.png}
\end{center}

D'autres technologies optionnelles secondaires ( 1\% du projet ) ont aussi servi sur Gouraya Quiz , notamment: 
\begin{itemize}
\item ajax
\item bootstrap
\end{itemize}
\begin{center}
\includegraphics[width=50pt]{ajax.png}
\includegraphics[width=50pt]{bootstrap.png}
\end{center}

\section{La structure du site}
  Notre site Gouraya Quiz possède une structure assez simple , avec 2 interfaces possibles et 2 types de connexion.
  La structure se présente comme suit:
  \begin{center}
\begin{tikzpicture}[sibling distance=20em,
  every node/.style = {shape=rectangle, rounded corners,
    draw, align=center,
    top color=white, bottom color=blue!20}]]
  \node {Accueil}
    child { node {Inscri.Etudiant} 
    	child { node {Forumulaire.Inscri.} }
    }
    child { node {connexion}
      child { node {Etudiant}
      	child { node {Répondre quiz}
        	child { node {Liste des quizz} 
        		child { node {Classement} }  }
        }
   	  }
   	 child { node {Enseignant}
   	 	child { node {Création quiz}
        	child { node {Liste des quiz} 
        		child { node {Classement} } } }
       }};
\end{tikzpicture}
\end{center}

\section{Les fonctionnalités du site}
\subsection{Inscription compte Étudiant}
Contrairement aux enseignants , n'importe quel étudiant peut s'inscrire et créer un compte sur Gouraya Quiz.
Il suffit de renseigner son nom, son adresse mail et un mot de passe via un formulaire.
Biensûr les champs de notre formulaire sont soumis à une vérification. L'email doit être valide ( de type "xxx@xxx.xxx" ).

\includegraphics[width=350pt]{inscription.png}
 
\subsection{Interface Enseignant}
Après authentification et connexion à son compte , un enseignant se voit offrir la possibilité de:
\begin{center}
\begin{itemize}

\item voir la liste des questionnaires déjà crées:
\newline
\includegraphics[width=350pt]{liste_quiz_prof.png}
\item supprimer des questionnaires , mais aussi en ajouter de nouveaux:
Chaque questionnaire porte un titre.
\newline
\includegraphics[width=350pt]{ajouter_quiz.png}
\item A la création d'un questionnaire , il est capable d'ajouter le nombre de questions qu'il veut , de définir un score pour chaque question. il peut associer le nombre d'options qu'il souhaite à son questionnaire. Il devra bien évidemment préciser l'option qui est une réponse correcte.
\newline
\includegraphics[width=350pt]{ajouter_question.png}
\item  Il peut aussi accéder au classement de ses étudiants et donc de voir leurs résultats.
\end{itemize}
\end{center}




\subsection{Interface Étudiant}
Après inscription , et connexion à son compte , un étudiant peut:
\begin{center}
\begin{itemize}

\item accéder aux différents questionnaires crées par un enseignant.
\newline
\includegraphics[width=350pt]{liste_quiz_etudiant.png}
\newline
\item  Il peut répondre au questionnaire de son choix.
le QCM s'affiche dans une fenêtre au design sobre et épuré , avec l'énoncé de la question , les options de réponse ( boutons de type radio ) , un bouton pour valider sa réponse et un autre pour retourner à la liste des questions du QCM.
\newline
\newline
\includegraphics[width=350pt]{qcm.png}
\newline
\newline
\item Quand un QCM est fait par l'étudiant son statut 'répondu' passe au vert.
\newline
\newline
\includegraphics[width=350pt]{repondu.png}
\newline
\newline

\item voir sa note ( son score ) ainsi que son classement dans un tableau comparatif le classant par rapport aux autres étudiants inscrits.
\newline
\newline
\includegraphics[width=350pt]{classement.png}
\newline
\end{itemize}
\end{center}
	

\section{La base de données et la partie PHP/MYSQL}
\subsection{Objet PDO}
Gouraya Quiz se sert de la base de données fournie sur Webetu.
Nous la connectons via un objet PDO afin de pouvoir nous en servir au fur et à mesure et garantir une expérience optimale à l'utilisateur. 
Il faut dire que la base de données est la pièce incontournable du puzzle. Tout y est stocké ( identifiants comptes enseignants,
identifiants comptes étudiants , les questions ...etc)
\subsection{Structure de la base de données}
Notre base de données contient 5 tables :

\subsubsection{Table étudiant:} 
Un étudiant est caractérisé par :
\begin{itemize}
\item un identifiant etudiant (id\_etudiant).
\item un nom (nom).
\item un email (email).
\item un mot de passe(password\_etudiant).
\item une note(score).
\end{itemize}

\subsubsection{Table prof\_auth:} 
Un enseignant est caractérisé par :
\begin{itemize}
\item un email (email).
\item un mot de passe(password\_etudiant).
\end{itemize}


\subsubsection{Table quiz:}
Un questionnaire est caractérisé par :
\begin{itemize}
\item un titre (titre\_quiz).
\item un identifiant (id\_quiz).
\end{itemize}


\subsubsection{Table question} 
Une question est caractérisée par :
\begin{itemize}
\item un identifiant(id\_question).
\item un titre (titre\_question).
\item une liste des réponses possibles (liste\_options).
\item la réponse correcte (option\_juste)
\item le barème de la question (score)
\item un identifiant quiz qui relie la question à son questionnaire respectif (id\_quiz).
\end{itemize}

\subsubsection{Table reponse:}
la table reponse rassemble les précédentes tables. Ses champs sont :
\begin{itemize}
\item id\_quiz.
\item id\_question.
\item id\_étudiant.
\item score\_attribuer.
\end{itemize}






 
\subsection{Manipulation de la base de données}
 A chaque fois qu'une questionnaire est crée par un enseignant, les tables quiz,question sont modifiées en conséquence. 
 Ceci est fait avec des requêtes SQL correspondantes.
 Lorsqu'un étudiant réponds à un questionnaire, c'est la table réponse qui est manipulée.
  
\section{Répartition du travail et Organisation:}
 	Nous avons réalisé la quasi totalité du projet ensemble. Mais nous avons essayé de répartir les tâches de la manière suivante 
  \begin{itemize}
   \item la partie HTML : MENAA Abderrezak.
   \item la partie CSS : MEZIANE Ghilas.
   \item le reste: PHP MYSQL.. : Travail en binôme.
  \end{itemize}
 \subsection{Difficultés rencontrées}
  Nous n'avons pas rencontré de grandes difficultés mais nous pouvions néanmoins optimiser notre travail. Faute de temps, 
  nous nous sommes contentés de faire de notre mieux pour respecter les consignes ainsi que les délais.
  Nous pouvions par exemple mieux organiser notre code, et utiliser des include afin de réduire la taille du code et l'optimiser. 


\end{document}
